\documentclass[conference]{IEEEtran}
\usepackage{pdftexcmds}
\usepackage[pdftex]{graphicx}
\usepackage{multirow}
\usepackage{pgfplots}
\usepackage{tikz}
\usepackage{balance}
\usepackage{amssymb, marvosym}
\usepackage{threeparttable}
\usepackage[bookmarks=false]{hyperref}
\usepackage{url}
\PassOptionsToPackage{hyphens}{url}
\hypersetup{colorlinks=true,breaklinks=true}
\usetikzlibrary{patterns,shapes,arrows}
\newcommand\tab[1][.5cm]{\hspace*{#1}}
\hyphenation{op-tical net-works semi-conduc-tor}
\IEEEoverridecommandlockouts
\raggedbottom
\newcommand{\tool}{tool-recommender-bot }
\begin{document}

% Copyright
%\setcopyright{acmlicensed}

\title{\tool}

\author{\IEEEauthorblockN{Chris Brown and Emerson Murphy-Hill}
\IEEEauthorblockA{Department of Computer Science\\
North Carolina State University\\
Raleigh, NC\\
Email: dcbrow10@ncsu.edu, emerson@csc.ncsu.edu}
}

\maketitle
\begin{abstract}
Recommendation systems were developed to improve the adoption of useful software tools and features designed to save time and effort in completing tasks that are often ignored by users. Previous research suggests that peer-to-peer recommendations are the most effective mode of tool discovery and that the receptiveness of recommendees is the most important characteristic in determining the outcome of tool recommendations. To help increase awareness of useful tools, we developed and evaluated a new recommendation system \tool designed to integrate aspects of peer interactions and user receptivity into automated tool suggestions for software developers of real-world applications. Our findings suggest that \tool is awesome, cool, and very effective in improving tool discovery.
\end{abstract}

\begin{IEEEkeywords}
Software Engineering; Tool Recommendation; Tool Discovery; Open Source
\end{IEEEkeywords}

\section{Introduction}
%Importance
Tool discovery is a problem... \\

Automated recommendation systems can help solve this problem... \\

But existing recommendations systems are ineffective... \\

Peer interactions and receptiveness are effective~\cite{vlhcc17}... \\

We created \tool to solve this... \\

\noindent
\textbf{RQ1:} How often can we expect \tool to make recommendations?  \\
\textbf{RQ2:} How useful are recommendations from \tool to developers?  \\

To answer these questions, we conducted a study analyzing \tool on five? popular open source Java projects to observe how many tool suggestions would be made based on past changes to the code base and how software developers reacted to receiving recommendations. This research makes the following contributions:\\

\begin{itemize}
\item We introduce the design and implementation of a novel automated recommendation system \tool
\item We provide implications for future
\end{itemize}

\section{Related Work}

Improving tool discovery...\\

Existing automated tool recommendation systems...

\section{Tool}
\tool is awesome. Here's how...

\subsection{Implementation}
\tool technical details...

\subsubsection{Jenkins}

\subsubsection{Error Prone}

\subsection{Receptiveness}
\tool was designed to integrate characteristics of peer interactions into automated recommendations. To better understand user-to-user recommendations and why they are effective for tool discovery, we observed how colleagues recommend tools to each other while completing tasks in prior work. Our results found that the receptiveness of users was the only significant indicator of determining whether a tool recommendation was effective or not compared to other interaction characteristics, such as politeness~\cite{vlhcc17}. Fogg defines receptiveness using two criteria, \textit{demonstrating a desire} and \textit{familiarity} with the adopted behavior~\cite{FoggPersuasive}.

\paragraph{Desire}

\paragraph{Familiarity}

\section{Methodology}

\subsection{Projects}

To evaluate the effectiveness of our recommendation system, we implemented \tool on five real-world open-source software applications. We selected projects hosted on Github\footnote{https://github.com}, a popular source code management site that hosts thousands of code repositories online. To narrow down projects for our evaluation, we picked Github projects that met the following criteria:

\begin{itemize}
\item one of the top Trending projects\footnote{https://github.com/trending} on GitHub based on activity by the community at the time of this writing,
\item primarily written in the Java programming language\footnote{https://java.com},
\item owned by a Github organization instead of personal user accounts,
\item and used Maven\footnote{https://maven.apache.org/} for software build management.
\end{itemize}

\begin{table}
	\centering
	\caption{Evaluation Projects}
	\begin{tabular}{|l|l|l|c|}
		\hline
		\textbf{Project} & \textbf{Java Files} & \textbf{LOC} & \textbf{Pull Requests} \\
		\hline
		spring-boot\footnote{https://github.com/spring-projects/spring-boot.git} &  &  & \\
		\hline
		dubbo\footnote{https://github.com/alibaba/dubbo.git} &  &  & \\
		\hline
		guava\footnote{https://github.com/google/guava.git}  &  &  & \\
		\hline
		retrofit\footnote{https://github.com/square/retrofit.git}  &  &  & \\
		\hline
		rocketmq\footnote{https://github.com/apache/rocketmq.git}  &  &  & \\
		\hline
		\end{tabular}
	\begin{tablenotes}
        \item[1] Details on projects used for study including GitHub repository name, number of Java files, lines of Java code, and total pull requests.
    \end{tablenotes}
	\label{tools}
\end{table}

Table~\ref{tools} presents the projects used for evaluating \tool in our study and provides details about each repository. These repositories span a wide range of organizations and real-world software applications, including a system to create ``stand-alone, production-grade'' Spring applications~\cite{spring-boot}, a high-performance remote procedure call framework from Chinese e-commerce company Alibaba~\cite{dubbo}, a collection of core Google libraries~\cite{guava}, an HTTP client for Android financial services mobile application Square~\cite{square-retrofit}, and a distributed messaging and data-streaming platform by Apache~\cite{rocketmq}.

\subsection{Study Design}

We divided our study into two segments to address each research question:

\subsubsection{RQ1}

Last 100 pull requests on repositories...

\subsubsection{RQ2}

Followed up with pull request authors to gather data on recommendation...

\section{Results}

\subsection{How often can we expect \tool to make recommendations?}

Tons of recommendations... \\

No false positives... \\

\subsection{How useful are recommendations from \tool to developers?}

Excellent responses from recommendees...\\

Statistically significant data...

\section{Discussion}

\subsection{Observations}

\subsection{Implications}

Here's what our results say about ways to improve tool recommendation systems...

\section{Limitations}

Internal\\

An external threat to the validity of our study is that we only observed open source projects hosted on Github in our evaluation. Our results may not generalize to closed source software projects and their developers. To minimize this, we selected popular real-world software applications on Github owned by organizations to avoid the use of personal development projects. Additionally, our recommendation system has limited generalizablility due to the fact we currently only assess recommendations for the Error Prone static analysis tool on Java projects that build with Maven. Future work will look to extend \tool to include different types of tools, programming languages, and build systems.

\section{Future Work}

More tools to recommend (static analysis, security, etc.) \\

More programming languages instead of just java...\\

More build systems (ant, gradle, TravisCI, bazel)...\\

\section{Conclusion}

\tool is awesome

%\balance
%\section{Acknowledgments}

%Thanks to all of the student and professional data analysts who volunteered for this study.

% The following two commands are all you need in the
% initial runs of your .tex file to
% produce the bibliography for the citations in your paper.
\bibliographystyle{abbrv}
\bibliography{paper}  
% You must have a proper ''.bib'' file
%  and remember to run:
% latex bibtex latex latex
% to resolve all references
%
% ACM needs 'a single self-contained file'!

%% That's all folks!
\end{document}
