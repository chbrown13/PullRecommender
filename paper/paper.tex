\documentclass[conference]{IEEEtran}
\usepackage{pdftexcmds}
\usepackage[pdftex]{graphicx}
\usepackage{multirow}
\usepackage{pgfplots}
\usepackage{tikz}
\usepackage{balance}
\usepackage{amssymb, marvosym}
\usepackage{threeparttable}
\usepackage[bookmarks=false]{hyperref}
\usepackage{url}
\PassOptionsToPackage{hyphens}{url}
\hypersetup{colorlinks=true,breaklinks=true}
\usetikzlibrary{patterns,shapes,arrows}
\newcommand\tab[1][.5cm]{\hspace*{#1}}
\hyphenation{op-tical net-works semi-conduc-tor}
\IEEEoverridecommandlockouts
\raggedbottom
\newcommand{\tool}{tool-recommender-bot }
\begin{document}

% Copyright
%\setcopyright{acmlicensed}

\title{\tool}

\author{\IEEEauthorblockN{Chris Brown and Emerson Murphy-Hill}
\IEEEauthorblockA{Department of Computer Science\\
North Carolina State University\\
Raleigh, NC\\
Email: dcbrow10@ncsu.edu, emerson@csc.ncsu.edu}
}

\maketitle
\begin{abstract}
Automated recommendation systems are important for improving tool discovery and increasing user productivity. Here's ours...
\end{abstract}

\begin{IEEEkeywords}
Software Engineering; Tool Recommendation; Tool Discovery; Open Source
\end{IEEEkeywords}

\section{Introduction}
%Importance
Tool discovery is a problem... \\

Automated recommendation systems can help solve this problem... \\

But existing recommendations systems are ineffective... \\

Peer interactions and receptiveness are effective\cite{vlhcc17}... \\

We created \tool to solve this... \\

\noindent
\textbf{RQ1:} How often can we expect \tool to make recommendations?  \\
\textbf{RQ2:} How useful are recommendations from \tool to developers?  \\

Contributions...\\

\section{Related Work}

Existing automated tool recommendation systems...

\section{Tool}
\tool is awesome. Here's how...

\subsection{Implementation}
Technical details...

\subsection{Receptiveness}
How we integrated receptiveness...

\paragraph{Desire}

\paragraph{Familiarity}

\section{Methodology}

\subsection{Projects}

Trending open source java projects on Github that build with maven used for evaluation...

\subsection{Study Design}

We divided our study into two segments to address each research question:

\subsubsection{RQ1}

Last n pull requests on repositories...

\subsubsection{RQ2}

Followed up with developers to gather responses...

\section{Results}

\subsection{How often can we expect \tool to make recommendations?}

Number of recommendations... \\

0 false positives... \\

\subsection{How useful are recommendations from \tool to developers?}

Response rate...\\

Qualitative responses... \\

\section{Discussion}

\subsection{Implications}

What do results say about ways to improve tool recommendation systems...

\subsection{Threats to Validity}

\subsection{Future Work}

\section{Conclusion}



%\balance
%\section{Acknowledgments}

%Thanks to all of the student and professional data analysts who volunteered for this study.

% The following two commands are all you need in the
% initial runs of your .tex file to
% produce the bibliography for the citations in your paper.
\bibliographystyle{abbrv}
\bibliography{paper}  
% You must have a proper ''.bib'' file
%  and remember to run:
% latex bibtex latex latex
% to resolve all references
%
% ACM needs 'a single self-contained file'!

%% That's all folks!
\end{document}
